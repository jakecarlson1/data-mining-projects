\documentclass{article}

\begin{document}

\title{Project 1}
\author{Jake Carlson}
\date{September 17, 2017}
\maketitle

\abstract
In this report, I will be examining how the United States federal government changed under President George W. Bush and President Barack Obama. I will look into the campaign platforms for each president, determine expectations for how the federal government would resize to meet each president's campaign promisses, and then look at actual policy decisions made by each president and how they changed the federal payroll. I will also examine natural disasters and geopolitical events that occured during each presidency and how the federal government reacted.
\newpage

\tableofcontents
\newpage

\section{Business Understanding}
For this report, I will be examining federal payroll data obtained by BuzzFeed News using the Freedom of Information Act. While the data set itself dates back to 1973, I will only be examining the data relevant to the presdency of George Bush (2001 - 2008) and Barack Obama (2009 - 2014). Although Obama's presidency lasted from 2009 to 2016, the data (at the time of writing this report) is only available through to 2014.
\par
Since Bush is a Republican, and Obama a Democrat, we would expect a dichotomy in their policy decisions. Generally, Republican candidates run a campaign focused on shrinking the federal government and lowering taxes, while Democratic candidates tend to emphasize coordination with allies and.. These are common trends for each party, but presidential candidates know that winning the presidency means winning party leadership, so each candidate will support changes in policy that draw new voters and reform their respective parties. Let's look more closely at the history of each candidate and their campaign platforms.

    \subsection{Political Careers}
    Each president had to build up a strong political career before they were considered by their parties as viable presidential candidates. Let's look briefly at the life and political history of each president.

        \subsubsection{George W. Bush}
        George W. Bush was born in New Haven, Connecticut on July 6, 1946. His family moved to Odessa, Texas in 1948, and then to Midland, Texas in 1950. Bush began Elementary school in Midland and finished in Houston after the family moved there in 1959. Bush went to high school at Phillips Academy Andover in Andover, Massachusetts where he developed a love for American History. He went to college at Yale University where he majored in history with a concentration in European and American studies. After he graduated in 1968, he joined the National Guard. He then continued his education at Harvard University where he obtained his MBA. George Bush met Laura Welch in July 1977 and they were married on November 5, 1977. Their twin girls were born in November 1981.\cite{bushhistory}
        \par
        Bush began his political career by campaigning to be the Mayor of Odessa when he was 31. He lost the election, but was on the campaign trail soon after to assist his father's run for the presidency in 1988. He then assisted in his father's reelection campaign in 1982, but they were defeated by Bill Clinton. Bush returned to Texas where he decided to run for Governor. He focused his campaign on education, juvenile justice, and welfare policies. He won the campaign, and worked hard with Texas Democrats to follow through with his campaign promisses. His most notable legislative accomplishment was overhauling the Texas education system, adding more choice and competition to the system and setting new skill requirements for children. He also introduced tax cuts, programs to help faith-based organizations, and began providing social services through churches.\cite{bushhistory}
        \par
        He won reelection as governor in 1998 and coined the term "compassionate conservatism", a brand meant to unite Republican values of the free market and small government with social welfare. His success as governor made him a prime choice to run for the presidency in the eyes of the Republican party.\cite{bushhistory}

        \subsubsection{Barack Obama}
        Barack H. Obama II was bown in Hawaii on August 4, 1961. His parents divorced and, after his mother remarried, he moved to Indonesia with his mother and stepfather. His mother was concerned about his education so she sent him back to Hawaii to live with his grandparents. He attended Punahou School from fifth grade through the end of high school. He then went to college, first at Occidental in Los Angeles, and then at Columbia University in New York City. He majored in political science and graduated in 1983. Obama then moved to Chicago to work as a community organizer. He organized the citizens of Altgeld Gardens to pressure Chicago's city hall to improve the living conditions in the public housing project. Unsatisfied with his success, he resolved to get a law degree. He attended Harvard Law School in 1988 and graduated magna cum laude. He met Michelle Robinson shortly after graduation and they were married in 1992. They had two daughters born in 1998 and 2001.\cite{obamahistory}
        \par
        Obama first ran for a political office in 1996, when he ran to be a state senator in Illinois. Despite being a member of the minority party in the state legislature, he was able to work with Republicans and Democrats to pass campaign finance reform and crime legislation. He became a leading legislator after the Democrats won Senate majority and passed nearly 300 bills focusing on helping children, old people, labor unions, and the poor. He then set his sights on a 2004 race for a U.S. Senate seat. He differentiated himself by opposing Bush's war in Iraq. He won the Senate seat by the largest margin in the history of Senate elections in Illinois.\cite{obamahistory}
        \par
        Obama's stance against the war set him apart from other potential Democratic presidential candidates. In a keynote address he gave at the Democratic National Convention in 2004, he struck notes of unity between political parties and different ethnicities. He promoted optimism with a phrase he borrowed from Reverend Jeremiah Wright, "the audacity of hope".\cite{obamahistory}

    \subsection{The Bush Presidency}
    Every presidency starts out with a campaign for office. A successful candidate makes a number of promises over the course of the campaign in hopes of getting voted into power. It is important that candidates to tailor their policies to their constituencies, so they don't drive away the coalition that votes their party into positions of power. The campaign is also an opportunity for candidates to present a vision for the future of their party. Let's look at the campaign that won Bush the presidency, some of the political policies we would expect to see to accomplish these promises, and some of the major events that occured during his presidency that altered how he managed the government.

        \subsubsection{The 2000 Presidential Campaign}
        The 2000 presidential campaign was a time to reform the Republican party. After the Clinton presidency, the Republican party wanted to move away from unpopular positions such as being opposed to all government programs. Bush wanted a party that worked to improve education, expand health care, and increase funding for Social Security and Medicare. He wanted the party to calm their rhetoric on morial issues such as abortion. Bush also spoke about cutting taxes for all Americans, while Democratic candidate Al Gore proposed tax cuts only as part of broader plans to implement policy. The two candidates also differed on how to combat teen pregnancy. Democrats wanted to implement sex education programs in public schools, while Republicans depended on abstinence by teenagers.\cite{bushcampaign2000}
        \par
        Bush was also a climate change denier. Having made a substantial profit when Bush exploration merged with Spectrum 7 in 1984, he had a vested interest in denying climate change and he made it part of his campaign platform.\cite{bushhistory}
        \par
        With these campaign promises, we would expect several areas of the government to get an increase in funding. The Department of Education should see an increase in funding to implement Bush's proposed education program, and the Department of Health and Human services should see an increase in funding to expand health care. However, if Bush implements his tax cuts, we should see funding cuts for a variety of other agencies to offset the loss in revenue the government pulls in. One such organization could be the Internal Revenue Service. We could also see cuts to the EPA in an effort to curb government support for researching climate change.

        \subsubsection{The Presidency}
        Bush wins the electoral college in 2000 despite falling short of achieving a majority of the popular vote. This leads Democrats to contest the result of the election, calling for recounts in several contested Florida counties. By the end of 2000, the Supreme Court voted to stop the recounts, and Gore conceded, leaving Bush President-elect.
        \par
        After his inauguration in January, 2001, Bush is quick to reinstate a ban on providing aid to organizations that perform abortions. He then moves to abandon the ratification of the Kyoto Protocol, which charged 180 countries to set limits on idustrial emissions. In June, Bush signs a \$1.35 trillion tax cut into law that reduces taxes for all income brackets and moves to eliminate the estate tax. Bush also outlines a new policy on stem cell research, where funding is continued for existing research, but bans the collection of additional stem cells.\cite{bushevents} It is typical for first-month presidents to begin enforcing policies that were not enforced by the previous leadership. This gives the appearance that they are enthusiatically getting to work to deliver on campaign promises.
        \par
        On September 11, 2001, the United States is attacked by terrorists. These individuals, working for the extremist group Al Qaeda, hijacked four airplanes and crashed two into the World Trade Center and one into the Pentagon. The fourth was suspected to be headed for the White House, but the passengers fought back againsts the hijackers and diverted the plane off course. It is the deadliest attack on American soil since the Japanese attacked Pearl Harbor. In an address to the nation, Bush vows to bring the group responsible to justice. With in the month, Bush appoints Pennsylvania governor Tom Ridge to a cabinet-level position to oversee the Office of Homeland Security. Ridge is tasked with coordinating efforts across over fourty federal agencies to protect the United States against future terrorist attacks.\cite{bushevents}
        \par
        Trailing his education reform in Texas as governor, in early 2002, Bush signs into law an education reform bill reinstating the No Child Left Behind policy. The bill provides more flexibility to local authorities on how they allocate funds and required new standardized tests for math and reading.\cite{bushevents}
        \par
        Bush's National Security Advisor, Condoleezza Rice, is questioned by Congress for information about warning signs before September 11th. In response, Bush announces sweeping changes to the security departments, placing the Office of Homeland Security in charge of over 100 agencies responsible for protecting the United States.\cite{bushevents} This marks a change in domestic policy that will likely result in additional funding for various security departments. The pressure to make this change originated from Congress, meaning the group responsible for the budget also supports this consolidation of power.
        \par
        Despite the ongoing war in Iraq, Bush continues to sign tax cuts into law, signing the third largest tax cut in history into law in May, 2003. In June of 2005, the Senate passes an energy bill meant to support the oil and natural gass energy industries, while providing tax incentives for the use of alternative energy sources. July of 2005 sees the launch of space shuttle Discovery, the first launch since the space shuttle Columbia failed during atmospheric reentry in 2003.\cite{bushevents}
        \par
        In August of 2005, Hurricane Katrina ravages the Gulf coast and severely damages the city of New Orleans. The levee systems meant to protect the city from flooding failed, causing massive destruction. Rather than allocating FEMA funds to the disaster, the White House is mute on the issue. The administration is harshly criticized for failing to use emergency funds to support aid the city.
        \par
        Bush approves a 700 mile long fence along the U.S.-Mexico border on October 26, 2006. We expect to see a spike in funding for Customs and Border Security around this time.
        \par
        The DOW Jones industrial average begins a sharp decline after an all-time high of 14,164 on October 9, 2007. As the housing crisis begins to set in, Bush proposes a stimulus package to encourage individuals to spend money and stimulate the slowing economy. The Senate votes to pass a slightly smaller stimulus package a few months later, providing Americans with tax rebates and businesses with tax breaks. The federal government resorts to taking over Fannie Mae and Freddie Mac in September, 2008 in an attempt to protect more than half of the countries mortgages. Bush then signs in a \$700 billion bailout plan to protect failing bank assets, the largest in history. He follows up with a bailout of General Motors and Chrysler to keep the two auto manufacturers a float shortly before leaving office. \cite{bushevents}

    \subsection{The Obama Presidency}

        \subsubsection{The 2008 Presidential Campaign}

        \subsubsection{The Presidency}


\section{Data Understanding}

\section{Data Preparation}

\section{Modeling}

\section{Evaluation}

\section{Conclusion}

\newpage

\begin{thebibliography}{10}
    \bibitem{bushhistory}
    Gary L. Gregg II
    \textit{George W. Bush: Life Before the Presidency}
    \texttt{https://millercenter.org/president/gwbush/life-before-the-presidency}

    \bibitem{obamahistory}
    Michael Nelson
    \textit{Barack Obama: Life Before the Presidency}
    \texttt{https://millercenter.org/president/obama/life-before-the-presidency}

    \bibitem{bushcampaign2000}
    Gerald M. Pomper
    \textit{The 2000 Presidential Election: Why Gore Lost}
    \texttt{https://www.uvm.edu/~dguber/POLS125/articles/pomper.htm}

    \bibitem{bushevents}
    University of Virginia - Miller Center
    \textit{George W. Bush - Key Events}
    \texttt{https://millercenter.org/president/george-w-bush/key-events}

\end{thebibliography}

\end{document}
