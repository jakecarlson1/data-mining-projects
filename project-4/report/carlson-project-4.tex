\documentclass{article}
\usepackage{graphicx}
\usepackage{hyperref}
\usepackage{multirow}

\begin{document}

\title{Project 4}
\author{Jake Carlson}
\date{November 24, 2017}
\maketitle

\abstract
Federal payroll data is used to perform clustering analysis. Both k-means and hierarchical clustering are used. First, k-means is used to cluster employees into groups of employees with similar qualities. We will see that supervisors tend to be middle-aged, and other groups form for employees older and younger than supervisors that are not in management positions. Second, hierarchical clustering is used to group agencies based on employee attributes. The goal was to find groupings that have similar functions, but additional data is needed for this to be a useful model.

\newpage

\tableofcontents
\newpage

\section{Business Understanding}
This report will focus on clustering analysis of the federal payroll data obtained by BuzzFeed News through the Freedom of Information Act. Specifically, k-means find groups of employees with similar attributes in the federal government and hierarchical clustering will be used to find agencies which have similar functions in the federal government.
\par
Armed with an understanding of what groups exist within each agency, agency leaders can work to create employee teams that are balanced with respect to the features each group holds. We can also see what features create the largest separation between the subgroups.
\par
With a model of what agencies influence and enforce similar parts of policy, a quick reference can be made to find agencies related to an agency of interest. It would also be useful to find what attributes make agencies appear to be related. If agency leaders know what other agencies are structured like theirs, they can reach out to these agencies when facing restructuring or other organizational changes.

\section{Data Preparation}
I will prepare my data in a similar fashion to Project 3 \cite{proj3}. I will take the middle of the age range for the age value of each employee. I will do the same thing for length of service. Education will be converted to an integer representing the number of years needed to achieve the degree the employee holds so that all education values are ratio scaled. I will also make supervisory status binary where a one indicates an employee is a supervisor and a zero indicates an employee is not a supervisor. The final list of attributes to be used for clustering is given in Table \ref{tab:1}. Previous projects have shown that Age, Education, Length of Service, Pay, and Supervisory Status are the most relevant employee attributes.
\par
All of these attributes are then scaled. With all of these attributes scaled, I can use Euclidean distance as my distance metric for clustering because all of the attributes are on the same scale.

    \begin{center}
        \begin{table}
            \centering
            \begin{tabular}{ |c|c|c| }
                \hline
                Attribute & Scale & Values \\
                \hline
                Age & Interval & 17, 22, 27, 32, 37, 42, 47, 52, 57, 62, 67, 72, 75 \\
                Education & Ratio & 1 - 22 \\
                LOS & Interval & 1, 3, 7, 12, 17, 22, 27, 32, 35 \\
                Pay & Ratio & 1 - 401,000 \\
                SupervisoryStatus & Nominal & 0, 1 \\
                \hline
            \end{tabular}
            \caption{Final Data Set Attributes}
            \label{tab:1}
        \end{table}
    \end{center}

\section{Modeling}
I will use two clustering methods to compare the structure of the government under Bush and Obama. First I will use k-means clustering. Then I will use hierarchical clustering.

    \subsection{K-Means}
    K-means attempts to fit the data set by placing k cluster centers and moving them until they cease to move by an amount larger than a given tolerance. It is required that you specify the number of centers to use before the algorithm begins. I will start by first using $k = 3$. I believe this will form groups that represent upper management, middle management, and entry level positions.
    \par
    The location of the cluster centers for the 2005 data set is given in Figure \ref{fig:1}. Here, cluster 2 represents supervisors where the average Pay and Education for the group tends to be higher. The other two clusters represent non-supervisors. Cluster 1 represents employees who are newer and younger, as indicated by the lower Length of Service and Age values. Their Education and Pay is lower than that of cluster 2. Cluster 3 represents employees who are older and have worked for the government for longer, but are not supervisors. Their education is higher than cluster 1, but lower on average than cluster 2. These groupings indicate that employees in the middle age range have a higher likelihood of being supervisors than older employees. Supervisors also tend to have gone to school for longer than non-supervisors.
    \par
    The cluster center locations for the 2013 data set is given in Figure \ref{fig:2}. Again, there is one cluster, cluster 3, that represents the supervisors. Again, the supervisors have an age in the middle of the other two cluster centroids. Cluster 1 represents young employees with some education who are new to working for the federal government. Cluster 2 represents older employees who are not supervisors. This cluster centroid is placed at the lower end of the education axis. I would expect more of the older employees to have a higher level of education, so there is a possibility that the centroid for Cluster 2 is getting stuck in an awkard position. To examine this further, I will rerun the clustering for 2013 using $k = 5$ to see if I can get clusters that are more representative of the underlying distribution of employees.
    \par
    The cluster centroids for 2013 with $k = 5$ are given in Figure \ref{fig:3}. We see that the oldest employees have been split by two clusters. In Cluster 4, employees have a slightly higher education on average and have worked for the government for much longer than employees in Cluster 5. In this clustering, Cluster 3 contains the youngest employees, and Cluster 1 contains employees in the middle of the age range who have the highest levels of education.
    \par
    It still feels like there are more groups hidden in these clusters. To examine how many groups is appropriate for k-means, I will run the algorithm for a varying number of of groups and look at the within sum of squares. By plotting these values and looking for the sharpest turn, we can determine the optimum value for the number of groups to use to cluster this data set using k-means. This plot is given in Figure \ref{fig:4}. The sharpest turn is where $k = 8$, however, the slope increases fairly consistently over the number of clusters, indicating there is not a single number of clusters that emerges as best for this data set. This is likely due to a lack of structure in the data, and must be explored further.
    \par
    The within cluster sum of squares for $k = 3, 5, 8$ is given in Table \ref{tab:2}. This shows that as the number of clusters increases, the variation within a cluster decreases. This makes sense, as increasing the number of centroids will make it easier for points to sit closer to an existing centroid. However, it is easier to understand the data if there are fewer clusters. That is why I would recommend using three clusters for k-means, as this produces clusters with easy to understand and identifiable employee attributes.

    \begin{center}
        \begin{figure}
            \includegraphics[scale=0.4]{./images/3-cluster-center-2005.pdf}
            \caption{The Location of Cluster Centers for 2005 where k = 3}
            \label{fig:1}
        \end{figure}
    \end{center}

    \begin{center}
        \begin{figure}
            \includegraphics[scale=0.4]{./images/3-cluster-center-2013.pdf}
            \caption{The Location of Cluster Centers for 2013 where k = 3}
            \label{fig:2}
        \end{figure}
    \end{center}

    \begin{center}
        \begin{figure}
            \includegraphics[scale=0.4]{./images/5-cluster-center-2013.pdf}
            \caption{The Location of Cluster Centers for 2013 where k = 5}
            \label{fig:3}
        \end{figure}
    \end{center}

    \begin{center}
        \begin{figure}
            \includegraphics[scale=0.4]{./images/wss-2013.pdf}
            \caption{Within Sum of Squares for 2013}
            \label{fig:3}
        \end{figure}
    \end{center}

    \begin{center}
        \begin{table}
            \centering
            \begin{tabular}{ |c|c| }
                \hline
                k & Within Sum of Squares \\
                \hline
                3 & 42.6\% \\
                5 & 59.3\% \\
                8 & 69.3\% \\
                \hline
            \end{tabular}
            \caption{Within Cluster Sum of Squares for 2013}
            \label{tab:2}
        \end{table}
    \end{center}

    \subsection{Hierarchical Clustering}
    In hierarchical clustering, a hierarchy of clusters can be formed by joining neighboring clusters in sequence. I will use complete link clustering, which assigns each data point its own group and joins the closest groups together. Here, closest is defined as the Euclidean distance between two points, or cluster centroids. For this process, I want to cluster agencies, so I will perform some additional processing to prepare agency data. I will take the mean Age, Education, and Pay for each of the three group identified using k-means. The goal here is to find clusters of agencies which carry out similar functions in the federal government.
    \par
    Looking at the dentrogram for 2005, given in Figure \ref{fig:4}, it would appear that there are 8 groups that form. The agencies are projected onto two principal components in Figure \ref{fig:5} with the clusters drawn for visualization. Some of the groups contains agencies that are related. For example, Group 5 has the Office of Management and Budget, the Council of Economic Advisors, the Commodity Futures Trading Commission, the Federal Trade Commission, the Securities and Exchange Commission, and the Office of the U.S. Trade Representative. These agencies all have to do with financial regulation and policy, but this group also contains the National Science Foundation and the Office of Science and Technology Policy. These two agencies are also related to each other, but they have little to do with the financial regulators. This grouping worked out fairly well. Unfortunately, the remaining groups have little to do with the purpose of each agency. For example, Group 4 has the Arctic Research Comission, the Nuclear Waste Techinical Review Board, the Marine Mammal Commission, the national Council on disability, the Federal Mine Safety and Health Review Commission, and the Medicare Payment Advisory Commission. These agencies are loosely related at best.
    \par
    Looking for agencies AN, AW, BK, BW, CX, FK, GO, GY, MA, NK, OS, RS, ZL (Group 4) in the first quarter of Figure \ref{fig:6} shows why this group is separated. These agencies have employees with slightly higher education levels, more pay, and that are older. This graph reveals the main issue with clustering agencies this way. The problem here is that the selected attributes for each agency represent features of employees, and have little to do with the structure or functionality of the agency.
    \par
    It would be necessary to incorporate additional data about each agency that describes features such as funding, organization, the number of appointed positions, and some description of what each agency does for the nation. A more robust clustering could potentially be obtained by examining the agency names and clustering by common words, but this clustering would be crude.
    \par
    The attributes I choose are not great for clustering agencies because all of the agencies are similar. Agencies have a similar distribution of education and age. Also, all of the agencies follow the General Schedule which clearly defines how much an employee is to be paid based on their attributes.
    \par
    This clustering also determined some agencies to be outliers. Groups 7 and 8 only contain one agency each, and they fall far away from the remaining agencies. The Commission on Review of Overseas Military Structure (YA, Group 8) has average pay and education, but much older employees than the rest of the agencies. The International Boundary Commission: U.S and Canada (GX, Group 7) has lower paid and younger employees than all of the remaining agencies by a substantial margin.
    \par
    To try to see if there is a more suitable number of groups that could be used to cluster this data, I plotted the average silhouette width for clusters while varying the number of clusters from 2 to 16. This plot is given in Figure \ref{fig:7}. The peak at $k = 2$ indicates the non-separability of the data. This graph would also indicate that my selection of 8 groups was flawed, and I should have opted for using six or four groups instead. Though, this simply would have joined neighboring groups and would not have produced clusters that are more representative of the various functions of the federal government.

    \begin{center}
        \begin{figure}
            \includegraphics[scale=0.4]{./images/2005-dentrogram.pdf}
            \caption{Agency Dentrogram for 2005 with 8 Groups Highlighted}
            \label{fig:4}
        \end{figure}
    \end{center}

    \begin{center}
        \begin{figure}
            \includegraphics[scale=0.4]{./images/2005-cluster-plot.pdf}
            \caption{2005 Agency Clusters Projected onto Two Principal Components}
            \label{fig:5}
        \end{figure}
    \end{center}

    \begin{center}
        \begin{figure}
            \includegraphics[scale=0.4]{./images/2005-pca.pdf}
            \caption{2005 Agency Clusters Projected onto Two Principal Components}
            \label{fig:6}
        \end{figure}
    \end{center}

    \begin{center}
        \begin{figure}
            \includegraphics[scale=0.4]{./images/2005-sil-width.pdf}
            \caption{2005 Average Silhouette Width}
            \label{fig:7}
        \end{figure}
    \end{center}

    \begin{center}
        \begin{figure}
            \begin{enumerate}
                \item Group 1: AMERICAN BATTLE MONUMENTS COMMISSION, FED MEDIATION AND CONCILIATION SERVICE, INTERNAT BOUNDARY \& WATER CMSN: US \& MEX, GOVERNMENT PRINTING OFFICE, PEACE CORPS, OFC OF NAVAJO AND HOPI INDIAN RELOCATION, ARMED FORCES RETIREMENT HOME, SELECTIVE SERVICE SYSTEM, JAPAN-UNITED STATES FRIENDSHIP CMSN, ANTITRUST MODERNIZATION COMMISSION
                \item Group 4: AFRICAN DEVELOPMENT FOUNDATION, ARCTIC RESEARCH COMMISSION, JAMES MADISON MEMORIAL FELLOWSHIP FOUND, NUCLEAR WASTE TECHNICAL REVIEW BOARD, NAT CMSN ON LIBRARIES AND INFO SCIENCE, FARM CREDIT SYSTEM INSURANCE CORPORATION, VIETNAM EDUCATION FOUNDATION, INTERNATIONAL JOINT CMSN: U.S. \& CANADA, MARINE MAMMAL COMMISSION, NATIONAL COUNCIL ON DISABILITY, OCCUPATIONAL SAFETY \& HEALTH REVIEW CMSN, FED MINE SAFETY AND HEALTH REVIEW CMSN, MEDICARE PAYMENT ADVISORY COMMISSION
                \item Group 5: FEDERAL LABOR RELATIONS AUTHORITY, MERIT SYSTEMS PROTECTION BOARD, DEFENSE NUCLEAR FACILITIES SAFETY BOARD, OFFICE OF MANAGEMENT AND BUDGET, COUNCIL OF ECONOMIC ADVISERS, COMMODITY FUTURES TRADING COMMISSION, COUNCIL ON ENVIR QUAL/OFC OF ENVIR QUAL, FEDERAL TRADE COMMISSION, FEDERAL HOUSING FINANCE BOARD, HARRY S. TRUMAN SCHOLARSHIP FOUNDATION, MILLENNIUM CHALLENGE CORPORATION, NATIONAL SCIENCE FOUNDATION, SECURITIES AND EXCHANGE COMMISSION, OFFICE OF THE U.S. TRADE REPRESENTATIVE, OFFICE OF SCIENCE AND TECHNOLOGY POLICY
                \item Group 6: CHRISTOPHER COLUMBUS FELLOWSHIP FOUNDATN, BARRY GOLDWATER SCHOL \& EXCEL IN ED FOUN, HELP ENHANCE LIVELIHOOD OF PEOPLE CMSN
                \item Group 7: INTERNAT BOUNDARY CMSN: U.S. AND CANADA
                \item Group 8: CMSN ON REV OF OVERSEAS MIL STRUCTURE
            \end{enumerate}
            \caption{Agencies in Each Hierarchical Cluster for 2005}
            \label{tab:3}
        \end{figure}
    \end{center}

\section{Evaluation}

    \subsection{K-Means}
    Using k-means to cluster employees showed that this data was not particularly well suited for clustering. Each time the data was clustered, supervisory status was a major factor affecting the cluster centroids. For both three and five groups, a single centroid moved to include supervisors. The remaining centroids divided the non-supervisors by the remaining available traits.
    \par
    Both the 2005 and 2013 clusterings showed that supervisors tend to be in the middle of the age range on average. The Obama administration appeared to promote more loyal employees to supervisory positions more often, as seen by the higher average Length of Service for supervisors than under Bush.
    \par
    We saw that increasing the number of clusters helped decrease the distance from points to their centroids, but this is not particularly useful given the non-separability of the data. For that reason, I recommend using fewer clusters to get a better idea of the types of employees on average. It is still useful to have this model to understand what traits are more common in supervisors than non-supervisors. This modeling could be applied to any particular agency and the groupings would be more indicative of the different types of employees. This model could help management target training programs for particular groups of employees, and potentially identify employees who look like they could fit into management.

    \subsection{Hierarchical Clustering}
    When clustering agencies based on employee attributes, hierarchical clustering did not form groups of agencies that had similar functions. A few of the clusters had agencies that are related, but the clustering was not robust. Some agencies that had more employees with higher education were grouped together, such as Group 5 in Figure \ref{tab:3} which had a lot of agencies that deal with financial and scientific regulation and policy.
    \par
    Additional data is needed about what each agency does in order for a robust model to be developed. With information about how each agency is structured and what effects they have on the nation, clusters could be formed that group agencies more on what they do.
    \par
    Until this additional data is generated or obtained, I would not recommend using this model as a way to interpret the structure of the federal government.

\begin{thebibliography}{10}
    \bibitem{proj1}
    Jake Carlson
    \textit{CSE 5331 - Data Mining Project 1}
    \texttt{https://github.com/jakecarlson1/data-mining-projects/blob/master/}
    \texttt{project-1/report/carlson-project-1.pdf}

    \bibitem{proj2}
    Jake Carlson
    \textit{CSE 5331 - Data Mining Project 2}
    \texttt{https://github.com/jakecarlson1/data-mining-projects/blob/master/}
    \texttt{project-2/report/carlson-project-2.pdf}

    \bibitem{proj3}
    Jake Carlson
    \textit{CSE 5331 - Data Mining Project 3}
    \texttt{https://github.com/jakecarlson1/data-mining-projects/blob/master/}
    \texttt{project-3/report/carlson-project-3.pdf}

\end{thebibliography}

\end{document}
